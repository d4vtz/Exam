%Exam Template CONAMAhh
%created by: David Torrez Reyes
\documentclass[a4paper, 12pt]{article}


% Package
\usepackage{amsmath,amssymb}
\usepackage{graphicx}
\usepackage{fancyhdr}
\usepackage[dvipsnames]{xcolor}
\usepackage[document]{ragged2e}
\usepackage{enumitem}
\usepackage[version=4]{mhchem}
\usepackage{siunitx}
\usepackage{chemfig}
\usepackage{multicol}
\usepackage{xparse}
\usepackage{derivative}

% Geometry
\usepackage[
%a4paper,            % Tamaño de hoja A4
includehead,        % Incluir el encabezado al calcular el margen
headheight=80pt     % Empezamos el texto a 50pt del borde superior 
]{geometry}
\geometry{top=0.2cm, bottom=1.0cm, left=1cm, right=1cm}


\pagestyle{fancy}
\fancyhf{}
\fancyhead[L]{\includegraphics{Images/logo.pdf}}
\renewcommand{\headrulewidth}{0pt}

\title{}
\author{
    \textbf{MATERIA:} \textcolor{red}{Cálculo Integral}\\
    % \textbf{PERIODO DE CLASES:} \textcolor{red}{29 de mayo al 9 de junio del 2023}\\
    \textbf{Profesor (a):} \textcolor{red}{David Torrez Reyes \hspace*{4cm} Clave: DV}\\
    \textbf{Alumno (a): \hspace*{10cm}}
    \vspace*{0.5cm}\\
    \textbf{Examen final}
}
\date{}


\NewDocumentCommand\Instructions{om}{
      \IfNoValueTF{#1}
      {#2}
      {
            #2
            \begin{flushright}
                  \textbf{$\mathbf{#1}$ puntos por reactivo}
            \end{flushright}
      }
}

\NewDocumentCommand\Question{mmmm}{
      #1
      \begin{enumerate}[label={\alph*)}]
            \item #2
            \item #3
            \item #4
      \end{enumerate}
}

\NewDocumentCommand\Problem{O{5}m}{
      #2
      \vspace*{#1cm}
}


\newenvironment{Questions}
{\begin{enumerate}[label=\arabic*]}
            {\end{enumerate}}

\newenvironment{Parts}
{\begin{enumerate}[label=\Roman*]}
            {\end{enumerate}}




\begin{document}
\maketitle\thispagestyle{fancy}


\Instructions[4]{Contesta correctamente las siguentes cuestiones}

\begin{Questions}
      \item \Question{
            Calcula el diferencial de la función:
            \begin{equation*}
                  y = 4x^3 + 2x^2 - 6x + 5.
            \end{equation*}
      }
      {$(12x^2 - 4x + 6) \odif{x}$}
      {$(12x^2 + 4x - 6) \odif{x}$}
      {$(12x^2 - 4x - 6) \odif{x}$}

      \item \Question{
            Para calcular la raiz de un número de forma aproximada, se utiliza el diferencial de la función $f(x) = \sqrt{x}$, el cual esta dado por:
      }
      {$\frac{\odif{x}}{\sqrt{x}}$}
      {$\frac{\odif{x}}{2\sqrt{x}}$}
      {$\frac{2\odif{x}}{\sqrt{x}}$}

      \item \Question{
            Para hallar el area bajo la curva de la función $f(x)$ en el intervalo $[a, b]$ utilizando la suma de Riemann, se necesita particionar el intervalo, por lo cual, con rectangulos de base:
      }
      {$x_{i} = a + i \Delta x$}
      {$x_{i} = a + i \Delta f$}
      {$x_{i} = a + i \Delta b$}

      \item \Question{
            El area bajo una curva esta dado por la suma de Riemann:
      }
      {$\displaystyle\lim_{n \to \infty} \displaystyle\sum_{i=1}^{n} f(x_{i}) \Delta x_{i}$}
      {$\displaystyle\lim_{n \to \infty} \displaystyle\sum_{i=1}^{n} f(x) \Delta x$}
      {$\displaystyle\lim_{n \to \infty} \displaystyle\sum_{i=1}^{n} f(x_{i}) \Delta x$}
      \newpage
      \item \Question{
            Identifica cual es el teorema fundamental del Cálculo:
      }
      {$\displaystyle\int_{a}^{b} f(x) \odif{x} = F(b) - F(a)$}
      {$\displaystyle\int_{a}^{b} f(x) \odif{x} = F(a) - F(b)$}
      {$\displaystyle\int_{b}^{a} f(x) \odif{x} = F(b) - F(a)$}

      \item \Question{
            Determina la antiderivada de la función $f(x) = 5x^4 + 3x^2 + 8$.
      }
      {$x^5 + x^3 + 8x + c$}
      {$2x^3 + x^2 + x^8 + c$}
      {$4x^5 + 2x^4 + 8 + c$}

      \item \Question{
            Cual es el resultado de la siguente integral:
            \begin{eqnarray*}
                  \displaystyle\int \sqrt[3]{x} \odif{x}
            \end{eqnarray*}
      }
      {$\displaystyle\frac{3 x^{\frac{3}{4}}}{4} + c$}
      {$\displaystyle\frac{3 \sqrt[3]{x^{4}}}{4} + c$}
      {$\displaystyle\frac{3 \sqrt[4]{x^{3}}}{4} + c$}
      \item \Question{
            Cual es el resultado de la siguente integral:
            \begin{eqnarray*}
                  \displaystyle\int \left( 3x^3 -8x^4 + \displaystyle\frac{10}{3} \right) \odif{x}
            \end{eqnarray*}
      }
      {$\displaystyle\frac{3 x^{4}}{4} - 8x^5 + \displaystyle\frac{10x}{3} + c$}
      {$\displaystyle\frac{3 x^{4}}{4} - \displaystyle\frac{40 x^5}{5} + \displaystyle\frac{10x}{3} + c$}
      {$\displaystyle\frac{3 x^{4}}{4} - \displaystyle\frac{8x^5}{5} + \displaystyle\frac{10x}{3} + c$}

      \item \Question{
            Al realizar la integral $\displaystyle\int \displaystyle\frac{(3x - 8)^3}{5} \odif{x}$, se realiza un cambio de variable, de cual se trata:
      }
      {$u = 3x - 8$}
      {$u = \displaystyle\frac{3x - 8}{5}$}
      {$u = (3x - 8)^3$}
      \newpage
      \item \Question{
            Dada el reactivo anterior, cual es el resultado de la integral:
      }
      {$\displaystyle\frac{(3x - 8)^5}{15} + c$}
      {$\displaystyle\frac{(3x - 8)^4}{75} + c$}
      {$\displaystyle\frac{(3x - 8)^5}{5} + c$}

      \item \Question{
            Al realizar la siguente integral $\displaystyle\int \displaystyle\frac{\odif{x}}{2x -3}$ se propone como cambio de variable $u = 2x - 3$. Entonces cual es diferencial $\odif{u}$:
      }
      {$2 \odif{x}$}
      {$-3 \odif{x}$}
      {$\displaystyle\frac{\odif{x}}{2}$}

      \item \Question{
            Cual es resultado de la integral del reactivo anterior:
      }
      {$\displaystyle\frac{1}{2}\ln (2x -3) + c$}
      {$\displaystyle\frac{1}{3}\ln (2x -3) + c$}
      {$\displaystyle\frac{1}{2}\ln (\displaystyle\frac{2x-3}{2}) + c$}

      \item \Question{
            la integral $\displaystyle\int \tan(x) \odif{x}$ tambien se puede expresar como:
      }
      {$\displaystyle\int \displaystyle\frac{\sin(x)}{\sec(x)} \odif{x}$}
      {$\displaystyle\int \displaystyle\frac{\sin(x)}{\cos(x)} \odif{x}$}
      {$\displaystyle\int \displaystyle\frac{\cos(x)}{\sin(x)} \odif{x}$}

      \Question{
            Cual es la factorización de $x^2 - 7x + 12$
      }
      {$(x - 4)(x + 3)$}
      {$(x + 4)(x - 3)$}
      {$(x - 4)(x - 3)$}

      \Question{
            Dada la fracción $\displaystyle\frac{ x + 1}{x^2 - 7x + 12}$, si se requiere expresar en sus fracciones parciales, entonces como la podemos reescribir:
      }
      {$\displaystyle\frac{A}{x-4} + \displaystyle\frac{Bx}{x-3}$}
      {$\displaystyle\frac{A}{x+4} + \displaystyle\frac{B}{x-3}$}
      {$\displaystyle\frac{A}{x-4} + \displaystyle\frac{B}{x-3}$}
      \item \Question{
            Siguendo el reactivo anterior, cual es el sistema de ecuaciones a resolver:
      }
      {$\left\{
                  \begin{array}{ l }
                        A + B = 1 \\
                        -3A - 4B = 1
                  \end{array}
                  \right.$}
      {$\left\{
                  \begin{array}{ l }
                        A + B = 1 \\
                        -3A - 4B = 0
                  \end{array}
                  \right.$}
      {$\left\{
                  \begin{array}{ l }
                        A + B = 0 \\
                        -3A - 4B = 1
                  \end{array}
                  \right.$}

      \item \Question{
            Si resolvemos el sistema de ecuaciones anterior obtenemos:
      }
      {$A=5$ y $B=4$}
      {$A=5$ y $B=-4$}
      {$A=-5$ y $B=4$}

      \item \Question{
      Al realizar la integral definida $\displaystyle\int_{0}^{2\pi} \sin(2x) \odif{x}$ se realiza el cambio de variable $u=2x$, entonces como cambian los limites de integración.
      }
      {$\displaystyle\int_{0}^{4\pi} \sin(u) \odif{u}$}
      {$\displaystyle\int_{0}^{\pi} \sin(u) \odif{u}$}
      {$\displaystyle\int_{0}^{\frac{\pi}{2}} \sin(u) \odif{u}$}

\end{Questions}
\end{document}

